%-------------------------------------------------------------------------------
% LATEX TEMPLATE ARTIKEL
%-------------------------------------------------------------------------------
% Dit template is voor gebruik door studenten van de de bacheloropleiding 
% Informatica van de Universiteit van Amsterdam.
% Voor informatie over schrijfvaardigheden, zie 
%                               https://practicumav.nl/schrijven/index.html
%
%-------------------------------------------------------------------------------
%	PACKAGES EN DOCUMENT CONFIGURATIE
%-------------------------------------------------------------------------------

\documentclass{uva-inf-article}
\usepackage[english]{babel}

\usepackage{listings}
\usepackage{xcolor}

\definecolor{codegreen}{rgb}{0,0.6,0}
\definecolor{codegray}{rgb}{0.5,0.5,0.5}
\definecolor{codepurple}{rgb}{0.58,0,0.82}

\lstdefinestyle{mystyle}{
    commentstyle=\color{codegreen},
    keywordstyle=\color{magenta},
    numberstyle=\tiny\color{codegray},
    stringstyle=\color{codepurple},
    basicstyle=\ttfamily\footnotesize,
    breakatwhitespace=false,         
    breaklines=true,                 
    captionpos=b,                    
    keepspaces=true,                 
    numbers=left,                    
    numbersep=5pt,                  
    showspaces=false,                
    showstringspaces=false,
    showtabs=false,                  
    tabsize=2
}

\lstset{style=mystyle}

\usepackage[style=authoryear-comp]{biblatex}
\addbibresource{references.bib}

%-------------------------------------------------------------------------------
%	GEGEVENS VOOR IN DE TITEL, HEADER EN FOOTER
%-------------------------------------------------------------------------------

% Geef je artikel een logische titel die de inhoud dekt.
\title{Bug Hunting Expedition}

% Vul de naam van de opdracht in zoals gegeven door de docent en het type 
% opdracht, bijvoorbeeld 'technisch rapport' of 'essay'.
\assignment{Model Based Testing with AMS}
\assignmenttype{essay}

% Vul de volledige namen van alle auteurs in en de corresponderende UvAnetID's.
\authors{René Kok; Tim van Ekert; Donovan Schaafsma; Tessa van Lobbrecht}
\uvanetids{13671146; 13635565; 13656007; 11848944}

% Vul de naam van je tutor, begeleider (mentor), of docent / vakcoördinator in.
% Vermeld in ieder geval de naam van diegene die het artikel nakijkt!
\tutor{Theo Ruys | Axini}

% Vul hier de naam van je tutorgroep, werkgroep, of practicumgroep in.
\group{Group 5}

% Vul de naam van de cursus in en de cursuscode, te vinden op o.a. DataNose.
\course{Software Specification, Verification and Testing}
\courseid{5364SSVT6Y}

% Dit is de datum die op het document komt te staan. Standaard is dat vandaag.
\date{\today}

%-------------------------------------------------------------------------------
%	VOORPAGINA 
%-------------------------------------------------------------------------------

\begin{document}
\maketitle

%-------------------------------------------------------------------------------
%	FINDINGS
%-------------------------------------------------------------------------------

\section{Findings}
\subsection{Besto}
We found \textbf{1} bug for Besto.

\subsubsection{The wrong state returned when closing an opened door}
\textbf{In what state(s)/situation(s) does the bug occur?},\\
The situation is that the door is opened and then the user wants to close the door.\\
\textbf{What is the expected behavior?}\\
The door should be "closed".\\
\textbf{What behavior is observed instead?}\\
But the response is "opened".\\
\textbf{What is your hypothesisto  of how the bug works?}\\
The door returns the wrong state.

\subsection{Logica}
We found \textbf{1} bug for Logica.

\subsubsection{An invalid password should not unlock a door}
\textbf{In what state(s)/situation(s) does the bug occur?}\\
The situation is that the door is closed and locked, then the user wants to unlock the door with an invalid password.\\
\textbf{What is the expected behavior?}\\
The door should not open with an invalid password.\\
\textbf{What behavior is observed instead?}\\
The door does open with an invalid password.\\
\textbf{What is your hypothesis of how the bug works?}\\
The check if the password equals the set password is negated.

\subsection{OnTarget}
We found \textbf{1} bug for OnTarget.

\subsubsection{A password up to 9 digits should be allowed}
\textbf{In what state(s)/situation(s) does the bug occur?}\\
The situation is fourhat the program shouldn't allow more than 4 digits, but OnTarget's Smartdoor accepts up to nine digits as. \\
\textbf{What is the expected behavior?}\\
The program should fail because the maximum is four digits (so 9999).\\
\textbf{What behavior is observed instead?}\\
A password with the integer 999999999 is inserted, which should throw an \emph{invalid password} message.\\
\textbf{What is your hypothesis of how the bug works?}\\
The bug is that it is possible to insert up to 9 digits as a password, while this shouldn't be possible.

\subsection{Quickerr}
We found \textbf{1} bug for Quickerr.

\subsubsection{The door can be locked while it already is closed and locked}
\textbf{In what state(s)/situation(s) does the bug occur?}\\
The door is closed and locked.\\
\textbf{What is the expected behavior?}\\
The door can be unlocked and opened, but not be locked again.\\
\textbf{What behavior is observed instead?}\\
The system is locking the door again while it is closed and locked.\\
\textbf{What is your hypothesis of how the bug works?}\\
The bug is that a door can be locked over and over again while it already is locked.

\subsection{SmartSoft}
We found \textbf{1} bug for SmartSoft.

\subsubsection{Maximum amount of consecutive incorrect passwords should not be more than 3}
\textbf{In what state(s)/situation(s) does the bug occur?}\\
The situation is that a user is trying to put in the wrong password multiple times.\\
\textbf{What is the expected behavior?}\\
The system has to shut off after three times.\\
\textbf{What behavior is observed instead?}\\
The system is not shutting off after inputting the wrong password 3 times.\\
\textbf{What is your hypothesis of how the bug works?}\\
The door is accepting more than 3 consecutive incorrect passwords.

\newpage

\subsection{TrustedTechnologies}
We found \textbf{2} bugs for TrustedTechnologies.

\subsubsection{Passwords between 0 and 1000 should be allowed}
\textbf{In what state(s)/situation(s) does the bug occur?}\\
The situation is that the given password is lower than 1000 and is not accepted by the door.\\
\textbf{What is the expected behavior?}\\
The door should accept a password lower than 1000 and greater than 0.\\
\textbf{What behavior is observed instead?}\\
The door does not accept a password lower than 1000.\\
\textbf{What is your hypothesis of how the bug works?}\\
The password is checked if it’s larger than 1000 instead of 0.

\subsubsection{Door should response in 0.5 seconds}
\textbf{In what state(s)/situation(s) does the bug occur?}\\
The situation is that the returned response messages takes longer than 0.5 seconds.\\
\textbf{What is the expected behavior?}\\
The door should return response messages in 0.5 seconds..\\
\textbf{What behavior is observed instead?}\\
The door takes longer than 0.5 seconds to respond.\\
\textbf{What is your hypothesis of how the bug works?}\\
The door does return response messages in 0.5 seconds.


\subsection{UniverSolutions}
We found \textbf{1} bug for UniverSolutions.

\subsubsection{Only the incorrect password should add up to the incorrect password counter}
\textbf{In what state(s)/situation(s) does the bug occur?}\\
The situation is that the door shuts off on an invalid password.\\
\textbf{What is the expected behavior?}\\
The door should only shut off on incorrect password when trying to unlock. 
Not on invalid passwords when unlocking.\\
\textbf{What behavior is observed instead?}\\
The situation is that the door shuts off on an invalid password when trying to unlock.\\
\textbf{What is your hypothesis of how the bug works?}\\
The invalid passwords also counts towards the counter of the incorrect passwords. 
So, it will shut off after three times of invalid or incorrect input.


\subsection{XtraSafe}
We found \textbf{2} bugs for XtraSafe.

\subsubsection{An already opened door can be opened}
\textbf{In what state(s)/situation(s) does the bug occur?}\\
The situation is that the door accepts an 'open' command if the door is already open.\\
\textbf{What is the expected behavior?}\\
There should be an invalid command response since the door is already opened.\\
\textbf{What behavior is observed instead?}\\
The door is open and returns 'opened' on the 'open' command.\
\textbf{What is your hypothesis of how the bug works?}\\
There's no invalid command check on the command 'open' if the door is already open.

\subsubsection{An already closed door can be closed}
\textbf{In what state(s)/situation(s) does the bug occur?}\\
The situation is that the door accepts an 'close' command if the door is already closed.\\
\textbf{What is the expected behavior?}\\
The door can’t be closed again so it has to give an invalid command response.\\
\textbf{What behavior is observed instead?}\\
The door is trying to close again while it is already closed.\
\textbf{What is your hypothesis of how the bug works?}\\
The door is being closed while it is already closed and this is possible in the bug.

\newpage

%-------------------------------------------------------------------------------
%	APPENDIX 
%-------------------------------------------------------------------------------

\appendix 
\section{Smartdoor's Model}
\lstinputlisting[language=Ruby]{smartdoor.aml}

%-------------------------------------------------------------------------------
\end{document}